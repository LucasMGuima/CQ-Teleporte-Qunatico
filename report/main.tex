\documentclass[12pt, a4paper]{article}

% Pacotes úteis
\usepackage[utf8]{inputenc}
\usepackage[T1]{fontenc}
\usepackage[portuguese]{babel}
\usepackage{amsmath}
\usepackage{float}
\usepackage{amssymb}
\usepackage{graphicx}
\usepackage{hyperref} % Para links
\usepackage[left=2.5cm, right=2.5cm, top=3cm, bottom=3cm]{geometry} % Margens

% --- Comando para definir o caminho das imagens ---
% O caminho deve terminar sempre com uma barra (/)
\graphicspath{{../notebook/imgs/}}

\title{Estudo sobre o Teleporte Quântico}
\author{Lucas da Mata Guimarães}
\date{\today}

\begin{document}

\maketitle % Gera a página de título

\begin{abstract}
Este estudo apresenta os princípios fundamentais do teletransporte quântico, 
uma técnica que permite a transferência do estado quântico de uma partícula 
de um local para outro, sem a necessidade de deslocamento físico da matéria.
Apresentando três aplicações conceituais deste protocolo e o que podemos inferir
a partir destas em relação a viabilidade do teleporte quântico.
Discutindo qual a contribuição do \textbf{emaranhamento quântico} e as 
implicações do \textbf{Teorema No-Cloning} (Não-Clonagem) da computação quântica
trazem para o modo como transferimos informações.
\end{abstract}

\tableofcontents % Gera o sumário

\newpage % Começa o conteúdo principal em uma nova página

\section{Introdução ao Teletransporte Quântico}

O teletransporte quântico é um fenômeno fascinante da mecânica quântica, 
que se distingue da ficção científica por não envolver o teletransporte de 
matéria ou energia, mas sim a transferência de informação quântica, que 
nomeamos estado quântico, de um sistema para outro. Essa informação é 
codificada em qubits (bits quânticos).

\section{Princípios Fundamentais}

\subsection{O Qubit e o Estado Quântico}

Um qubit é a unidade básica de informação quântica, que pode ser um dos estados
$|0\rangle$ ou $|1\rangle$, ou estar em superposição, isto é, ter uma chance
$\alpha$ de estar no estado $|0\rangle$ e $\beta$ de estar no estado $|1\rangle$.
Formalmente representamos um estado quântico pela seguinte fórmula:

\begin{equation}
    \label{eq:estado_quantico}
    |\psi\rangle = \alpha|0\rangle + \beta|1\rangle
\end{equation} 

Onde $|\alpha|^2 + |\beta|^2 = 1$. O objetivo do teletransporte é transferir 
o valor de $(\alpha, \beta)$ de um local (Alice) para outro (Bob).

Quando encontramos um estado que possui a seguinte forma,

\begin{equation}
    \label{eq:estado_superposicao}
    |\psi\rangle = \frac{1}{\sqrt{2}}|0\rangle + \frac{1}{\sqrt{2}}|1\rangle 
\end{equation}

dizemos que temos um \textbf{estado em superposição}, isto é, quando medirmos o valor
deste estado, temos uma chance de lermos $|0\rangle$ ou $|1\rangle$.

\subsection{Mais de um Qubit}

Para representar um qubit, utilizamos a representação de estado como
$|0\rangle$ e $|1\rangle$, que mostra todos os estados possíveis desse
qubit, mas e quando trabalhamos com mais de um?

Tomemos como exemplo, utilizar dois qubits, assim, representamos seus
estados possíveis como, $|00\rangle$, $|01\rangle$, $|10\rangle$ e 
$|11\rangle$. Mostrando todas as combinações que podemos encontrar
esses dois qubits. De um modo geral, a equação do estado é
semelhante a equação \ref{eq:estado_quantico}:

\begin{equation}
    \label{eq:esqto_qunatico2qubit}
    |\psi\rangle = \alpha|00\rangle + \beta|01\rangle + \gamma|10\rangle + \delta|11\rangle
\end{equation}

Onde novamente, $|\alpha|^2 + |\beta|^2 + |\gamma|^2 + |\delta|^2 = 1$.

\subsection{Emaranhamento Quântico (Entanglement)}

O teletransporte quântico é impossível sem um par de partículas emaranhadas 
compartilhado entre os dois comunicadores, Alice e Bob. O estado emaranhado 
mais comum é o Estado de Bell $\Phi^+$:

\begin{equation}
    \label{eq:estado_bellPhiPlus}
    |\Phi^+\rangle = \frac{1}{\sqrt{2}}(|00\rangle + |11\rangle)
\end{equation}

Aqui, podemos ver que, mesmo trabalhando com 2 qubits, o a equação \ref{eq:estado_bellPhiPlus}
é menor que a equação \ref{eq:esqto_qunatico2qubit}. Isso é devido ao emaranhamento,
pois, os dois qubits assumem sempre o mesmo valor, isto é, os únicos
estados possíveis de se encontrar nosso sistema após o emaranhamento são
$|00\rangle$ e $|11\rangle$.

Este emaranhamento atua como um \textbf{canal quântico} para a transferência de informação.

\section{O Processo de Teletransporte}
O processo envolve três etapas principais:

\begin{enumerate}
    \item \textbf{Medição de Bell por Alice:} Alice realiza uma medição conjunta 
    (Medição de Bell) no seu qubit de entrada ($|\psi\rangle$) e em uma das partículas 
    do par emaranhado. Esta medição destrói o estado original $|\psi\rangle$ 
    (devido ao Teorema No-Cloning), mas codifica a informação necessária para a reconstrução. 
    O resultado é um dos quatro estados de Bell.
    \item \textbf{Comunicação Clássica:} Alice envia o resultado da sua medição 
    (2 bits de informação clássica) para Bob.
    \item \textbf{Reconstrução por Bob:} Bob aplica uma operação unitária de correção 
    (uma das quatro transformações de Pauli, $I, X, Y, Z$) na sua partícula emaranhada, 
    com base nos 2 bits de informação clássica recebidos, o que a transforma no estado 
    original $|\psi\rangle$.
\end{enumerate}

\section{Experimento}

Os experimentos a seguir, têm como intuito analisar a capacidade do teleporte quântico,
transferindo a informação de um qubit para outro. Foram realizados para cada experimento
duas simulações e uma execução utilizando a IBM Plataforma Quantum. A codificação foi
realizada utilizando a SDK Qiskit.

\subsection{Teleporte Quântico de 1 Qubit}

O seguinte circuito foi desenvolvido para analisar o protocolo de teleporte quântico
do estado de um qubit.

\begin{figure}[h!]
    \centering
    \caption{Circuito: Teleporte Quântico de 1 qubit}
    \includegraphics[width=1\textwidth]{circuito_teleport_1qubit.png}
    \label{fig:circ_1qubit}
\end{figure}

Na configuração apresentada, preparamos o qubit $Q$, com o estado $|1\rangle$,
e queremos passar esse estado para o qubit $Bob$, primeiro, vamos simular o
circuito.

Quando simulamos o circuito sem considerar o ruído encontramos o seguinte
cenário.

\begin{figure}[H]
    \centering
    \caption{Resultados na simulação sem ruído}
    \includegraphics[width=0.75\textwidth]{plot_1qubit_semRuido.png}
    \label{fig:plot_1qubit_semRuido}
\end{figure}

Agora quando levamos em consideração o ruído que encontramos em computadores
quânticos, temos os seguintes resultados na simulação.

\begin{figure}[H]
    \centering
    \caption{Resultados na simulação com ruído}
    \includegraphics[width=0.75\textwidth]{plot_1qubit_comRuido.png}
    \label{fig:plot_1qubit_comRuido}
\end{figure}

Quando consideramos o ruído, começamos a ver que em algumas de nossas medições
encontramos o estado $|0\rangle$ no qubit $Bob$, porém o ruído é tão pequeno
que o desconsideramos.

Para comparação, o circuito foi executado em um computador quântico disponibilizado
pela IBM, e os resultados encontrados se assemelham à simulação com ruído.

\begin{figure}[H]
    \centering
    \caption{Resultados na execução na Plataforma IBM Quantum}
    \includegraphics[width=0.75\textwidth]{plot_1qubit_ibm.png}
    \label{fig:plot_1qubit_ibm}
\end{figure}

Para as análises feitas, foram realizados $1000$ testes, para ter uma coleta relevante
e confiável da informação. Isso nos mostra que na simulação com ruído 
\ref{fig:plot_1qubit_comRuido}, medimos o estado $|1\rangle$ um total de $986$ vezes, 
e na execução em um computador quântico \ref{fig:plot_1qubit_ibm}, medimos o estado
$|1\rangle$ um total de $963$ vezes.

\subsection{Teleporte Quântico Superposição}

A computação quântica, tem como principais fundamentos a superposição quântica,
logo seria vantajoso conseguirmos fazer o teleporte deste estado.

Para isso, o seguinte circuito foi proposto.

\begin{figure}[H]
    \centering
    \caption{Circuito: Teleporte Quântico em Superposição}
    \includegraphics[width=1\textwidth]{circuito_teleport_superposicao.png}
    \label{fig:circ_superposicao}
\end{figure}

Na figura acima \ref{fig:circ_superposicao}, preparamos no qubit $Q$ o seguinte estado
de superposição,

\begin{equation}
    \label{eq:circ_superposicao}
    |\phi\rangle =  \frac{-i}{\sqrt{2}}|0\rangle + \frac{i}{\sqrt{2}}|1\rangle.
\end{equation}

A simulação desconsiderando o ruído do circuito \ref{fig:circ_superposicao},
serve para vermos se o circuito é teoricamente viável, e para este os resultados
encontrados foram.

\begin{figure}[H]
    \centering
    \caption{Resultados na simulação sem ruído}
    \includegraphics[width=0.75\textwidth]{plot_superposicao_semRuido.png}
    \label{fig:plot_super_semRuido}
\end{figure}

Os resultados nos indicam que o circuito é viavel, visto que, temos uma
distribuição de mais ou menos $50\%$ entre os estados $|0\rangle$ e $|1\rangle$,
que representa superposição desejada.

Quando levamos em consideração o ruído, encontramos o seguinte resultado.

\begin{figure}[H]
    \centering
    \caption{Resultados na simulação com ruído}
    \includegraphics[width=0.75\textwidth]{plot_superposicao_comRuido.png}
    \label{fig:plot_super_comRuido}
\end{figure}

Nesta situação, o ruído se mistura com os resultados possíveis, visto que
ao medirmos apenas um qubit os estados possíveis são apenas $|0\rangle$ e
$|1\rangle$.

Ao executarmos o circuito em um computador quântico, os resultados obtidos foram.

\begin{figure}[H]
    \centering
    \caption{Resultados na execução na Plataforma IBM Quantum}
    \includegraphics[width=0.75\textwidth]{plot_superposicao_ibm.png}
    \label{fig:plot_super_ibm}
\end{figure}

Assim como na análise com ruído \ref{fig:plot_super_comRuido}, o ruído
característico de um computador quântico, se mescla aos resultados possíveis
ao utilizarmos de um computador quântico real.

\subsection{Teleporte Quântico $N$ Qubits}

Para uma Comunicação ser viável precisamos passar mais de um bit ou qubit
por vez, para analisar essa situação, o seguinte circuito de teleporte 
quântico de 2 qubits foi elaborado.

\begin{figure}[H]
    \centering
    \caption{Circuito: Teleporte Quântico 2 Qubits}
    \includegraphics[width=1\textwidth]{circuito_teleport_nQubits.png}
    \label{fig:circ_nQubits}
\end{figure}

Neste circuito preparamos o estado $|01\rangle$ nos qubits de $Q$ para ser
teleportado para $Bob_{0}$ e $Bob_{1}$.

A viabilidade teórica desse circuito pode ser verificada na simulação sem
ruído, que apresentou os seguintes resultados.

\begin{figure}[H]
    \centering
    \caption{Resultados na simulação sem ruído}
    \includegraphics[width=0.75\textwidth]{plot_nQubits_semRuido.png}
    \label{fig:plot_nQubits_semRuido}
\end{figure}

Comprovando que podemos teoricamente teleportar o estado de 2 qubits.

Quando adicionamos o ruído que ocorre normalmente em computadores quânticos
os resultados encontrados são.

\begin{figure}[H]
    \centering
    \caption{Resultados na simulação com ruído}
    \includegraphics[width=0.75\textwidth]{plot_nQubits_comRuido.png}
    \label{fig:plot_nQubits_comRuido}
\end{figure}

Assim como visto nos casos anteriores onde adicionamos o ruído, ele adiciona
uma leve flutuação do valor, porém continua claro qual o resultado verdadeiro.
Uma ocorrência interessante nesta análise, é que o estado $|10\rangle$ nunca
ocorreu durante os testes executados.

A título de comparação, o circuito foi avaliado em um computador quântico
real, e os seguintes resultados foram encontrados.

\begin{figure}[H]
    \centering
    \caption{Resultados na execução na Plataforma IBM Quantum}
    \includegraphics[width=0.75\textwidth]{plot_nQubit_ibm.png}
    \label{fig:plot_nQubits_ibm}
\end{figure}

Assim como os resultados encontrados na simulação com ruído \ref{fig:plot_nQubits_comRuido},
aqui vemos que o estado $|10\rangle$ novamente não ocorre, e os resultados
são novamente semelhantes à simulação com ruído.

Esse fenômeno onde o estado $|10\rangle$ não ocorre tanto na simulação \ref{fig:plot_nQubits_comRuido},
quanto em um computador quântico real \ref{fig:plot_nQubits_ibm},
necessita de um estudo próprio, pode vir a ser relativo ao emaranhamento
entre os qubits $Alice_{0} \space Bob_{0}$ e $Alice_{1} \space Bob_{1}$,
mas um estudo mais aprofundado se faz necessário. 

\section{Conclusão}
As vantagens que os protocolos de comunicação quântica trazem ao jogo são a 
segurança e a eficiência (ou densidade de dados).

Na questão de segurança, o ganho advém do uso do emaranhamento quântico e 
do Teorema da Não-Clonagem.

Dentro do Teleporte Quântico, qualquer tentativa de medir ou interceptar 
o estado em trânsito colapsaria o estado quântico, destruindo assim a 
informação e alertando as partes envolvidas sobre a espionagem. Isso é 
usado como base para a Distribuição de Chaves Quânticas (QKD), que garante 
uma chave criptográfica mais segura.

Na Comunicação Densa, utiliza-se também um par de qubits emaranhados 
previamente compartilhados, para passar dois bits de informação clássica 
usando apenas um qubit entre as partes, explorando o par pré-compartilhado 
de qubits emaranhados.

Quando falamos de velocidade, podemos atrelar a eficiência ao protocolo.

No Teleporte Quântico, a velocidade de transferência do estado quântico é 
limitada pela velocidade de transmissão da informação clássica (os dois bits), 
necessária para que o receptor reconstrua o estado teleportado. 
O emaranhamento, de fato, não se limita pela velocidade da luz, 
sendo instantâneo. Mas o ponto forte do protocolo está na transferência 
fiel do estado quântico, crucial para a Internet Quântica, permitindo 
estender as redes quânticas sem o risco de perda de decoerência que 
acontece em repetidores clássicos.

Dentro da Comunicação Densa, temos a vantagem na densidade de dados, 
permitindo o envio do dobro de informação clássica (dois bits) usando o 
canal quântico de um qubit. Isso traz uma maior densidade de dados para a 
comunicação.

\newpage

\section*{Referências}
\addcontentsline{toc}{section}{Referências} % Adiciona as referências ao sumário
% Adicione suas referências aqui no formato desejado. Exemplo:

\begin{description}
    \item[1] Matthew Silverman, "Quantum Teleportation | PennyLane Demos", \\ \url{https://pennylane.ai/qml/demos/tutorial_teleportation}
    \item[2] IBM Quantum, "Quantum Teleportation", \\ \url{https://quantum.cloud.ibm.com/learning/pt/courses/basics-of-quantum-information/entanglement-in-action/quantum-teleportation}
    \item[3] Yanofsky, N. S.; Mannucci, M. A. \par Quantum Computing for Computer Scientists. Cambridge: Cambridge University Press, 2008.
\end{description}

\end{document}